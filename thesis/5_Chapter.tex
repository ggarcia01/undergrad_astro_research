\chapter{Discussion}
\label{chap5}

\section{Summary}
\label{sub5_1}

We have described the process for which we cross-correlated sources between the ETS, 2RXS, and CSC2.0, which allowed for a direct comparison of their 0.5-2.0 X-ray fluxes. 
Using a threshold of 7, we have constructed a sample of variable X-ray sources (Chapter 3). 
We optically investigated this sample through the SDSS and careful literature review.
From the 12 sources that were observed and had optical spectra in SDSS, we have determined that 5 AGN have experienced a dramatic change in their X-ray variability that is due to changes in their mass accretion rate. 
These 5 AGN all dimmed in brightness somewhere between Einstein and Chandra observations. 
On the other hand,  we concluded that the X-ray variability of the other t sources is not accretion related. 
The reasons for non-accretion related X-ray variability ranged from relativistic beaming of radio-loud galaxies to observed diffuse emission from galaxy clusters (Chapter 4).


\section{Big Picture}
\label{sub5_2}



Because changes in the accretion state of AGN are events that occur on cosmic timescales, they are rare events to observe given our limited baseline provided by the history of imaging X-ray Astronomy. 
The results presented here reinforce that since, in the end, we find only five AGN that potentially have had major changes in their accretion state. 
However, the limitation on our results is not solely due to the cosmic timescale of these events. 
The need to have multiple observations of the same source at different epochs is a major limiting factor on our results. 
We made use of observations from three X-ray telescopes: Einstein, ROSAT, and Chandra. 
Each has its own distinct footprint, which do not necessarily overlap extensively in some parts of the sky. 
The sources available for investigation for this project are those found in the intersection of their footprints. 
This limits extensively the amount of data that can be considered for this project. 
Another major cut is made to the data available for consideration when looking for optical follow-up data in the SDSS. 
Without corresponding optical data, there is not much that can be done to investigate the nature of the source producing the X-ray emission. 
All these factors play a significant role in limiting our search for changes in the accretion state of AGN.

Nonetheless, our results motivate the continuation of the search of these types of objects.
Finding more of them and creating a more extensive sample, we can begin to find patterns that potentially {\it explain\/} the changes in accretion states. 
Specifically, we can begin to compare the optical morphologies of the variable sources to those of non-variable AGNs using high-resolution imaging. 
Over a considerable amount of time, we can potentially find trends in how galaxy morphologies are related to the activity in the galactic center.  
We can also begin to explore the environments of the galaxies and the environments around the galaxy centers. 
This can help identify patterns that link changes in accretion state to some property related to its host galaxy. In doing this, we are inspecting the conditions which directly affect BH fueling.

\section{Future Work}


We want to continue to monitor the status of the five AGN observed to potentially have a change in their accretion rate. 
We see that the last Chandra observations for all five sources occured in the early to mid- 2000s, more than fifteen years ago now for most of these sources.
A Chandra follow up conducted now will reveal their current X-ray flux standing, and an optical follow up will offer a modern spectroscopic classification of these objects. 
We want to follow a similar procedure for the variable X-ray sources that did not have an SDSS observation. 
This can increase our sample if more AGN with accretion-related X-ray variability are found. 
Along with adding new data to our existing sample of X-ray variable AGN, we can also look back at X-ray data archives and conduct an extensive search for serendipitous observations that could fill in the gaps between the Einstein, ROSAT, and Chandra observations. 
A more complete light curve can restrict the era in which the change in accretion states occurred. 
It will similarly help solidify these events as long-term processes.



\subsection{New Era of Sky Surveys}

With the continuous advances in technology, new telescopes that possess increasing capabilities will emerge.
Two such telescopes that will surely benefit a project like this are the extended R\"{o}ntgen Survey with an Imaging Telescope Array (eROSITA) X-ray Observatory and the Vera C. Rubin Observatory (previously known as the Large Synoptic Survey Telescope).

The eROSITA, which launched on July 13, 2019, will conduct an All Sky Survey (called the eROSITA All-Sky Survey or eRASS) in the 0.2-10 keV energy band in similar fashion to the RASS.
That is, eRASS will also scan the sky with a rotation axis in the plane of the Earth' orbit, which will create great circles that overlap at the ecliptic poles. 
In this 4 year survey, the full sky will be observed every 6 months, leading to 8 total passes of the sky in 4 year survey. 
In the 0.5-2.0 keV soft X-ray band, the eRASS will be roughly 20 times more sensitive than the RASS. 
This will result in an expected detection of 10,000 galaxy clusters and 3 million AGN, greatly increasing the population of known X-ray sources \citep{merloni2012}.

Optically, the Vera C. Rubin Observatory will conduct the Legacy Survey of Space and Time (LSST) over 10 years, which will cover 18,000$^{\circ}$ of southern hemisphere sky.
This will complement the SDSS's northern hemisphere survey. 
While the LSST is primarily focused on observing objects within the Milky Way, it is still projected to observe 20 billion galaxies \citep{ivezi2019}.

Future iterations of this project will have a larger amount of X-ray data and greater potential of optically identifying the sources because of these surveys. 
The eRASS and the LSST will provide an enormous amount of new data that will aid in exposing the true nature of black hole fueling. 
