\chapter{Introduction}
\label{chap1}


\section{AGN Phenomenon - Black Hole Paradigm}
\label{sub1_1}

It is now widely accepted that there exists a black hole at the center of most galaxies.
Because black holes are some of the most energy efficient objects in the universe, their accretion of mass plays a vital role in their host galaxy's evolution  \citep{Schawinski2012}.
Studies of samples of galactic centers demonstrate a strong correlation between black hole masses ($M_{\text{BH}}$) and host galaxy stellar velocity dispersions ($\sigma$), where $M_{\text{BH}} \propto \sigma^{4.02\pm 0.32}$ \citep{Tremaine2002}.
Similarly, \cite{KormendyAndHo2013} give further insight of a black hole's influence on its host galaxy by providing evidence of a linear relationship between the mass of the black hole and the luminosity of the bulge ($L)$.
Because of the mass-stellar velocity dispersion ($M_{\text{BH}}-\sigma$) and mass-luminosity ($M_{\text{BH}}-L$) relations, current theories suggest a black hole-galaxy co-evolution via galaxy mergers: mergers grow galaxies and their stellar bulges and facilitate the flow of gas to the center of the potential well where it can be accreted by the black hole \citep{haring2004} .
Therefore, for bigger galaxies, more massive super-massive black holes (SMBHs) are found at the center. 


A fraction of the galaxies that host SMBHs are actively accreting material; such galaxies are called active galactic nuclei (AGN).
In AGN, SMBHs have an accretion disk that we assume to be geometrically thin but optically thick.
The accretion disk emits radiation like a blackbody.
By the virial theorem, the accretion disk converts half of its gravitational potential energy into radiation and half into heating the disk.
Because of this, we find that most of the continuum emission comes from the accretion disk. 
The accretion disk is also responsible for producing high energy (ultraviolet and soft X-ray) photons that ionize the gas surrounding it.
There is a cycle of ionization and recombination that produce radiation in two distinct regions of an AGN.
These two regions give rise to optical emission lines. 
They are the broad line region (BLR) and the narrow line region (NLR).

The BLR is the innermost nebular structure of an AGN, found at a radius of less than $\sim$ 0.5 pc away from the SMBH.
Because the BLR has a lot of mass ($\sim$ 2000 M$_{\odot}$) concentrated in a tight space, the BLR has a density that is much larger than all the critical densities of the elements that compose the gas within the BLR.
Thus, there are many collisional excitations and de-excitations of electrons in metastable state and thus no forbidden lines are produced in the BLR.
Due to their proximity to the SMBH, gas clouds in the BLR have high velocity dispersion, which causes the permitted emission lines to have broad componenets with full-width at half-maximum (FWHM) of several thousand km s$^{-1}$.

In contrast, the NLR is located well beyond the BLR and the accretion disk.
The NLR is at a radii ranging from $\sim$10-100 pc.
While the NLR has a higher total mass ($10^6$M$_{\odot}$) compared to the BLR, it is also much larger in size, so the density of the NLR is significantly lower than the density of the BLR. 
In fact, it is lower than the critical densities of most elements that compose the gas within the NLR.
Thus, narrow forbidden lines dominate the emission from the NLR.

The classification of active galaxies is largely based on the emission-line features present in a galaxy's spectrum.
AGN with spectra that have a BL and NL components are type 1 Seyferts, while AGN with spectra that only have a NL component are classified as type 2 Seyferts \citep{khachikina1974}.




\section{Mass Accretion Rates to Observed Typical Luminosities}
\label{sub1_2}

As mentioned above, AGNs are powered by mass accretion onto the SMBH.
The conversion of accretion disk’s gravitational potential energy produces radiation in the form of high energy photons.
The intense radiation emits over a broad frequency range, including in the ultraviolet (UV) and X-ray bands.
But because of the varying physical conditions in AGN such as black hole masses and fueling processes, the luminosities produced vary from object to object depending on the mass accretion rate.
Thus, observed luminosities of AGN offer a different way to classify AGN.
Seyfert galaxies have typical bolometric luminosities of $\sim10^{44}$ erg~s$^{-1}$ \citep{Woo&Urry2002}.
There exists, however, a class of AGN called quasi stellar radio sources (quasars) that are much more luminous than typical Seyferts --- so much so that they dominate the luminosity coming from the galaxy, drowning out stellar emission from the stars in the host galaxy.
This contrasts from Seyferts in that the total luminosity is a more even split between the nucleus and the stellar emission. 
Typical bolometric luminosity values of quasars are order of $\sim$10$^{47}$ erg~s $^{-1}$ \citep{Woo&Urry2002}.

In order to produce such luminosities, mass accretion rates must be high, but this also implies that the mass available to be accreted is also high.
Therefore, the luminosity of an AGN is dependant on the mass of the black hole and the accretion rate and takes the following functional form:

\begin{equation}
    \text{L} = \eta  \dot{m}  c^2
\end{equation}

where $\eta$ is the radiative efficiency, $\dot{m}$ is the accretion rate, and $c$ is the speed of light.
 We can use this equation to derive a luminosity at the Eddington limit, which is the maximum luminosity possible before radiation pressure begins to dominate and push mass outwards.
This results in the following equation from \cite{peterson1997}:

\begin{equation}
    \text{L}_{ \text{Edd}  } = 1.38\times 10^{38} \Big ( \frac{ \text{M}_{  \text{BH} }  }{ \text{M}_{\odot}   } \Big ) \ \text{erg} \cdot \text{sec}^{-1} 
\end{equation}


Seyferts and quasars often reach luminosities close to the Eddington limit.
To do this, a lot of mass is neeeded.
Assuming a radiative efficiency of 10\% as in \cite{soltan1982}, 2.5 solar mass per year must be accreted for a SMBH  of typical mass 10$^8$ M$_{\odot}$ to reach the Eddington luminosity
It is currently unclear how exactly the fuel needed to accrete is made available. 



\section{The AGN Fueling Problem}
\label{sub1_3}

 One large scale process for how AGN are thought to be fueled is through in-falling matter from the host galaxy. Such matter would be gas and dust expelled from stellar winds, supernova, and interstellar clouds \citep{balick&heckman1982}.
The accretion rate is dependant on the angular momentum of the matter being accreted.
In order for the matter to fuel the AGN, its angular momentum must be low or dissipate over time so that the matter falls inward towards the center.
If the angular momentum is high, the matter will instead form an accretion disk around the SMBH, with a radius proportional to the magnitude of the angular momentum.

On a smaller scale, mass can also come from the halo gas that is tidally disrupted from dense star clusters near the galactic center \citep{frank1979}.
If a star passes within the Roche radius limit, a tidal disruption event occurs, in which the mass ejected is accreted onto the center.
Hence, the magnitude of the accretion rate will be proportional to the density of the nearby star clusters \citep{Hills1975}.
There have been observations of tidal disruption events in AGNs in the past 5-8 years.

In addition to in-falling matter, the galaxy merger hypothesis offers an alternative solution to how AGN are able to find the fuel necessary to produce their luminosities \citep{Sanders1996}.
 For gas and dust-rich (“wet”) mergers, lots of star-forming regions are formed due to collisions of molecular clouds.
The new-found matter slowly finds its was to the galactic center where it will become a new source of fuel to power the high luminosity AGN.
Because galaxies are close to each other relative to their diameters, mergers are relatively common.
Thus, simulations that mimic the observed galaxy number density give evidence in favor of AGN fueling and black hole accretion being primarily fueled by galaxy mergers \citep{DiMatteo2005}. 

\section{Episodic Accretion}
\label{sub1_4}

While there are various methods for how an AGN can be fueled, the accretion of an AGN is not uniform over time.
Observations of accretion rates over long periods of time have shown evidence of accretion behaving episodically where the AGN experiences periods of activity and quiescence \citep{SaikiaAndJamrozy2009}.
The episodic nature of the accretion poses the following questions: What is responsible for this phenomenon? 
Are there specific events that cause the AGN to run out and/or get resupplied with the fuel needed to grow its SMBH and produce luminosity?

Due to the Eddington limit, there is a maximum rate at which an AGN can accrete matter before radiation pressure becomes greater than the gravitational pressure. 
Above that point, accretion ceases because matter begins to be pushed away from the BH.
These AGN outbursts can distribute the gas in the galactic center and quench star formation.
Additionally, \cite{morganti2017} suggests that these outbursts occur in cycles and required in order to prevent cool gas from pilling up in the center of the galaxy. 
Using optical and radio AGN, \cite{morganti2017} traces signatures of past nuclear activity to constrain the period of these cycles of accretion and quiescence.

For optical AGN, a technique called “light echoes” makes it possible  to infer the timescale of an AGN’s most recent active period. 
This technique is applied to quiescent AGN with highly ionized gas clouds around them. 
Because the nucleus of the quiescent AGN is now not bright enough to excite the ionized gas, the ionized gas that is present must have been a result of the previous period of AGN activity. 
This leads to time estimates for the luminous episodes of an AGN of around $.2 - 2\times 10^5$ years \citep{morganti2017}.

In radio AGN, the radio emission is created from the synchrotron radiation from relativistic electrons. 
This results in a radio spectrum that results in a power law distribution. 
\cite{morganti2017} uses the fact that changes to the shape of the spectrum are caused when ageing the source and realizing that a steep break appears in the spectrum when nuclear activity stops and the electrons are no longer being replenished. 
Thus, modeling radio spectrum and its steepness can give estimates on the period of quiescent and active cycle.
The results suggests that the on/off cycles of radio AGN can have timescales of 10$^7$ to 10$^8$ years \citep{morganti2017}.


To gain more insight into the evolution of black holes and AGN over time, populations of AGN must be gathered and studied.
Perhaps, there is a connection between the accretion episodes of an AGN and the stages in its evolution or the evolution of its host galaxy.
It has been observed that there is no difference in quasar populations at low redshift compared to quasar populations at high redshifts \citep{Elvis1994}.
Therefore, it would be helpful to identify AGN at any redshift that have undergone dramatic changes in accretion, which could be revealed by dramatic changes in luminosity.


\section{Can We Find AGN That Turn On/Off?}
\label{sub1_5}

As mentioned, it is known that AGN must fuel in an episodic manner.
However, the timescales for these episodes are not well understood.
If an AGN whose accretion turns on or off is observed, it can provide a handle on a certain phase of evolution for the SMBH, which can be related to the fueling of the AGN.
To observe such an event, the AGN must experience dramatic variability in its accretion flows.
While low-amplitude variability is common on short time scales for quasars \citep{BlandfordAndMcKee1982}, dramatic variability over a large period of time is not common. 
Thus, observing an AGN turn on or off in its accretion is a rare event, so only a few instances of this have been recorded.
SDSS J015957.64+003310.5 was the first very luminous AGN witnessed to have dramatic change in its accretion \citep{LaMassa2015}.
Between 2000 and 2010, it varied so dramatically that the AGN changed spectroscopic classifications from a Type 1 quasar to Type 1.9 AGN.
In the X-ray, SDSS J015957.64+003310.5 experienced a flux drop by a factor of 7.2 in the 2-10 keV band \citep{LaMassa2015}.
AGN that experience dramatic changes in spectroscopic classification and luminosity are called "changing look" AGNs (CLAGNs).
There are, however, arguments against the change in spectroscopic classification being due to the change in accretion.
The change in classification can also be explained by a tidally disrupted massive star, which creates a flare with rise and decay times consistent with what was observed
\citep{merloni2015}.
To approach a more definitive answer on the nature of CLAGNs and the episodic fueling of them, a larger sample of AGN that turn on or off is required.

\section{Our Approach}
\label{sub1_6}

CLAGNs have largely been found by looking for variability using optical surveys. 
Along with variability in optical bands, AGN also experience dramatic variability in other bands like infrared and X-ray.
Thus, looking variability at these other wavelengths can be useful in identifying more CLAGNs.

For our approach, we choose to search for long-term, dramatic variability in the X-ray emission. 
While other bands also experience variability, X-ray emission is directly linked to the accretion of mass onto black holes, so X-ray emission can further provide a handle on the ways in which AGN are fueled.
Additionally, although the timescales of episodes of activity or quiescence of accretion are unknown, we do understand that they occur on cosmic timescales.
So, in order to maximize our detection rate, we also want to maximize the time baseline for our variability search as this will increase the likelihood of catching an AGN turn on or off event.
For that reason, we choose to search datasets from the Einstein X-ray Observatory and the Chandra X-ray Observatory, which provide data that span the roughly 40 year history of imaging X-ray astronomy (Chapter 2). 
We generate a catalog of variable X-ray sources (Chapter 3) and examine their optical properties to determine their activity status (Chapter 4). 
Ultimately, our sample of highly variable AGN may provide insight on the episodic nature of accretion and on the overall history of BH growth over cosmic time.


